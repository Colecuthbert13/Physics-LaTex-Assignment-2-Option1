%! program = pdflatex
%\documentclass[12pt,a4paper]{memoir} % for a long document
\documentclass[12pt,a4paper]{article}% for a short document

\usepackage{setspace}
\title{LaTex Assigment 2: \\Option 2}
\author{Cole Cuthbert}
\date{9/30/15} % Delete this line to display the current datex
\doublespacing
%%% BEGIN DOCUMENT
\begin{document}
\maketitle

	When considering the two cars and their paths around the track, there are several factors that are constant and the same for both cars while their velocity is $50{{meters}\over{seconds}}$. The various components of the movement of the two cars - radius$(r)$, angular acceleration$(\alpha)$, angular velocity$({\omega})$, radial velocity$(v_r)$ and radial acceleration$(a_r)$ can be seen in the total acceleration equation ${\vec{a}}={{r}{\alpha}{\hat{\theta}}-{r}{\omega^2}{\hat{r}}+{2}{\nu_r}{\omega}{\hat\theta}+{a_r}{\hat{r}}}$. The four components contain every variable effecting the cars and their movements. The first component $({r}{\alpha}{\hat{\theta}})$ stands for the tangential acceleration of the cars, and is found by multiplying the radius of the track ($r$) with the angular acceleration ($\alpha$). Angular acceleration is determined by the difference in the angular velocity or $\omega$ over a given amount of time in seconds $({{\alpha}={{\Delta\omega}\over{\Delta}{t}}})$. Angular velocity is determined by the change in angle over the change in time $({{\Delta}{\Theta}\over{\Delta}{t}})$. The ${\hat\theta}$ stands for a unit of angle in the positive direction. For these cars, the only time when tangential acceleration will be relevant will be during the passing movement because both cars have a constant angular velocity of $50m/s$ for the rest of the problem. The next component $({r}{\omega^2}{\hat{r}})$ represents the centripetal force on the cars. The centripetal force is calculated by the radius multiplied by the angular velocity to the second power $(r\times\omega^2)$. This force is constantly being applied in the radial direction to any object in a turn at any degree or speed and in the case of these cars, is constant except during the passing movement due to a change in speed and radius of the trailing car. The next component $({2}{\nu_r}{\omega}{\hat\theta})$ is called the Coriolis acceleration. This acceleration is the result of the doubling of the product of the radial velocity and the angular velocity $(2\times({v_r}\times{\omega}))$. This acceleration in the angular direction $(\hat\theta)$ compensates for the varying tangential velocities that exist as one changes their radius, thus keeping the object on track to its desired location. The final component is the radial acceleration $({a_r})$ and consists solely of the change in radial velocity over the change in time $({{\Delta{v_r}}\over{\Delta{t}}})$. This, such as the centripetal acceleration, is applied in the radial direction $(\hat{r})$ either towards or away from the center depending on the sign of the change in radial velocity. These are the four components of acceleration and they will not change unless one of the cars speeds up, slows down, changes direction, or any combination of the three, such as in a passing maneuver.
	
	If the second car in this example were to want to move ahead of the car 1 $meter$ in front of it, there would be three ways to accomplish that. Each method would include a sequence of three directional movements, the difference in technique being the order in which those steps take place. The first method would entail the rear car speeding up until it was directly behind the front car, slowing down to match the first car's speed, moving 2 $meters$ (the width of the cars) in the outward radial direction and then reaccelerating forward until it has passed the first car where is will turn left and travel 2 $meters$ towards the center (in the inwards radial direction). The second method would be if the rear car were to travel (accelerate) 2 $meters$ outwards in the radial direction first, then proceed to accelerate angularly and pass the first car, finally traveling 2 $meters$ in the inwards radial direction after having fully passed the once leading car. The third method would be a combination of the previous two, where the car would accelerate both angularly and radially with the car traveling 2 $meters$ radially outwards while closing the 1 $meter$ gap between the two cars. In each model, the four components of acceleration change during this passing maneuver, although depending on the method of passing, when and how much some components change is different despite each method achieving the same positional result (with the rear car in the lead). 
	
	In the first method, the tangential acceleration increases as a result of the change in angular velocity $(\omega)$ as the rear car closes the 1 $meter$ gap. Along with a change in tangential acceleration, the increase in angular velocity causes an increase in centripetal acceleration as it is the product of the radius and the angular velocity to the second power (${r}{\omega^2}$). Despite a change in angular velocity, the radius does not change as the gap is being closed in method 1 so neither the Coriolis nor the radial acceleration are effected. It isn't until that 1 $meter$ gap is closed, the car slows down and travels 2 $meters$ outwards radially that the two previously mentioned acceleration components change. With the 2 $meter$ change outwards, the radius of the second (rear) car is increase, affecting every single component in the acceleration equation. Although since the rear car slowed down to match the first cars speed before it travelled outwards, that must also be taken into account by reducing the angular acceleration down to 0 $m/s^2$. So despite the increase in radius, the tangential acceleration would become 0 $m/s^2$. After the car finishes traveling the 2 $meters$ outwards, it's radial acceleration is replaced by angular acceleration as the car speeds up once more and passes the first car. This change nullifies the Coriolis and radial acceleration until the rear car surpasses the first car and begins it's travel of 2 $meters$ inwards radially. The second method contains the exact same steps as method one, except the car travels the 2 $meters$ outward first and then stops radial movement and increases it's angular velocity until it has surpassed the first car. The third method is a combination of the two where each of the four acceleration variables are active at the same time, due to the fact that the car accelerates both angularly and radially instead of alternating back and forth. By accelerating both angularly and radially simultaneously, the car completes the maneuver quicker and more efficiently. It is for this reason that the third method of passing is the most favorable for the rear car although all three work and achieve the same goal of surpassing the first car.
	
	
\end{document}